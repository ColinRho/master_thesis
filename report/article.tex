%%% thesis
\documentclass[12pt]{article}\usepackage[]{graphicx}\usepackage[]{color}
%% maxwidth is the original width if it is less than linewidth
%% otherwise use linewidth (to make sure the graphics do not exceed the margin)
\makeatletter
\def\maxwidth{ %
  \ifdim\Gin@nat@width>\linewidth
    \linewidth
  \else
    \Gin@nat@width
  \fi
}
\makeatother

\definecolor{fgcolor}{rgb}{0.345, 0.345, 0.345}
\newcommand{\hlnum}[1]{\textcolor[rgb]{0.686,0.059,0.569}{#1}}%
\newcommand{\hlstr}[1]{\textcolor[rgb]{0.192,0.494,0.8}{#1}}%
\newcommand{\hlcom}[1]{\textcolor[rgb]{0.678,0.584,0.686}{\textit{#1}}}%
\newcommand{\hlopt}[1]{\textcolor[rgb]{0,0,0}{#1}}%
\newcommand{\hlstd}[1]{\textcolor[rgb]{0.345,0.345,0.345}{#1}}%
\newcommand{\hlkwa}[1]{\textcolor[rgb]{0.161,0.373,0.58}{\textbf{#1}}}%
\newcommand{\hlkwb}[1]{\textcolor[rgb]{0.69,0.353,0.396}{#1}}%
\newcommand{\hlkwc}[1]{\textcolor[rgb]{0.333,0.667,0.333}{#1}}%
\newcommand{\hlkwd}[1]{\textcolor[rgb]{0.737,0.353,0.396}{\textbf{#1}}}%

\usepackage{framed}
\makeatletter
\newenvironment{kframe}{%
 \def\at@end@of@kframe{}%
 \ifinner\ifhmode%
  \def\at@end@of@kframe{\end{minipage}}%
  \begin{minipage}{\columnwidth}%
 \fi\fi%
 \def\FrameCommand##1{\hskip\@totalleftmargin \hskip-\fboxsep
 \colorbox{shadecolor}{##1}\hskip-\fboxsep
     % There is no \\@totalrightmargin, so:
     \hskip-\linewidth \hskip-\@totalleftmargin \hskip\columnwidth}%
 \MakeFramed {\advance\hsize-\width
   \@totalleftmargin\z@ \linewidth\hsize
   \@setminipage}}%
 {\par\unskip\endMakeFramed%
 \at@end@of@kframe}
\makeatother

\definecolor{shadecolor}{rgb}{.97, .97, .97}
\definecolor{messagecolor}{rgb}{0, 0, 0}
\definecolor{warningcolor}{rgb}{1, 0, 1}
\definecolor{errorcolor}{rgb}{1, 0, 0}
\newenvironment{knitrout}{}{} % an empty environment to be redefined in TeX

\usepackage{alltt}
\usepackage[fleqn]{amsmath}
\usepackage{booktabs}
\usepackage{longtable}
\usepackage{natbib}
\usepackage{graphicx}
  \graphicspath{{../plot/}}
\usepackage{amssymb}
\usepackage{epstopdf}
\usepackage{color}
\usepackage{titlesec}
\newcommand{\cred}{ \color{red}}
\newcommand{\cgreen}{\color{green}}
\newcommand{\cblue}{\color{blue}}
\newcommand{\cmag}{\color{magenta}}
\newcommand{\bn}{\begin{enumerate}}
\newcommand{\en}{\end{enumerate}}
\newcommand{\bi}{\begin{itemize}}
\newcommand{\ei}{\end{itemize}}
\newcommand{\be}{\begin{eqnarray}}
\newcommand{\ee}{\end{eqnarray}}
\newcommand{\by}{\begin{eqnarray*}}
\newcommand{\ey}{\end{eqnarray*}}
\renewcommand{\labelenumi}{(\alph{enumi}) }
\newcommand{\beq}{\begin{equation}}
\newcommand{\eeq}{\end{equation}}
%
\usepackage[margin=2.2cm, includehead]{geometry}% see geometry.pdf on how to lay out the page. There's lots.
\geometry{letterpaper} % or letter or a5paper or ... etc
% \geometry{landscape} % rotated page geometry
%\bibpunct{(}{)}{;}{a}{,}{,}
%\setlength{\textwidth}{16cm}
%\setlength{\textheight}{21cm}

\newcounter{parnum}
\newcommand{\N}{%
  \noindent\refstepcounter{parnum}%
   \makebox[\parindent][l]{\textbf{[\arabic{parnum}]}}\quad  }
% Use a generous paragraph indent so numbers can be fit inside the
% indentation space.
\setlength{\parindent}{1.5em}

% See the ``Article customise'' template for come common customisations





\IfFileExists{upquote.sty}{\usepackage{upquote}}{}
\begin{document}

\title{Modelling the entire range of daily precipitation using mixture distributions}
\author{Hyunwoo Rho}
\date{}

\maketitle

\titlelabel{\thetitle.\quad}

%%%
%========================
\section{Introduction}
%========================


To measure and predict the precipitation are the most oldest concerns of mankind. It is an essential factor in agriculture, forest management, hydrology, and preparing for disasters like flood and drought. For generating stochastic precipitaion model, there are some feasible time intervals as a base period of modelling, but taking a daily basis is considerably natural \cite{richardson1981stochastic}. Constructing daily stochastic precipitaion model can be extended to precipitation analytics like monthly or yearly basis approach and spatial analysis. 


Many of previous studies tried to model daily rainfall data through parametric approaches e.g. \cite{ison1971wet, mielke1973three, richardson1981stochastic, stern1984model} and also non-parameteric approaches e.g. \cite{sharma1999nonparametric, harrold2003nonparametric}Along some studies in hydrology community, the evidence of heavy-tailed phenomenon on the distribution of high precipitation amount \cite{koutsoyiannis2004statistics}. In this perspective, various methods were introduced more focused on tail behavior of daily precipitation \cite{furrer2008improving, li2012simulation, papalexiou2012entropy, papalexiou2013extreme}. 


In this paper, we compare various type of parametric models and introdcue a particular type of mixture distribution called phase-type distribution. Through this model we tried to fit the whole range of continuous daily precipitation.


%%%
%========================
\section{Data Sets}
%========================


The main data set used to analyse is United States Historical Climatology Network(USHCN) Daily data set. Raw data set contains daily record of precipitation, snowfall, snow depth, maximum temperature, minumum temperature, and information about flag. Among 48 files of each sontiguous states, According as \cite{li2012simulation}, Texas was selected. For following their work, the same data selection criteria were adopted which are all nonzero precipation of 1940 to 2009 without taking care of missings. 


Additionally, Daily Global Historical Climatology Network(GHCN-DAILY) was used to do extra analyses \cite{papalexiou2012entropy, papalexiou2013extreme} which has identical data format with USHCN. This set contains about 100 thousand stations and also encompasses USHCN as its subset. By filtering stations along \cite{papalexiou2013extreme}, (a) record length of over 50 years, (b) percentage of missing values less than 20\%, data assigned with suspicious "quality flags" less than 0.1\%. The screen values of quality flags are two, one with "G"(failed gap check), and another with "X"(failed bound check). For more information about data set, see \cite{menne2012overview}. 


Finally we handled 49 stations in USHCN-Texas data, and 19328 stations in GHCN data after filtering. 



%%%
%========================
\section{Existing models}
%========================

	\subsection{One component models}


Fundamental characteristrics of daily precipitation data are described as follows; non-negative, continuous except at the spike on zero, right-skewed. Assuming we only focus on the continuous part of the data, such distributions include exponential \cite{todorovic1975stochastic}, gamma \cite{ison1971wet, wilks1999interannual, schoof2010development}, and kappa \cite{mielke1973three}, and so on. Let $X$ denote the nonzero daily precipitation amount, then each models can be presented as their probabilty density functions as below. 


\begin{equation}
  \label{exponential.pdf} 
  f(x ; \lambda) = \frac{1} {\lambda} e^{-x / \lambda}, \quad x \geq 0, \quad \lambda > 0,
\end{equation}


\begin{equation}
  \label{gamma.pdf} 
  f(x ; \alpha, \beta) = \frac {1} {\beta^\alpha \Gamma(\alpha)} x^{\alpha - 1} e^{-x/\beta} , \quad x \geq 0, \quad \theta > 0,
\end{equation}


\begin{equation}
  \label{kappa.pdf} 
  f(x ; \alpha, \beta, \theta) = \frac {\alpha \theta} {\beta} (\frac {x} {\beta})^{\theta - 1} [\alpha + (\frac {x} {\beta})^{\alpha \theta}]^{-\frac {\alpha + 1} {\alpha}}, \quad x > 0, \quad \alpha, \beta, \theta > 0,
\end{equation}

Other distributions like skewed-normal \cite{wan2005stochastic}, truncated power of normal distribution \cite{bardossy1992space}, and only for the tail of data, genralized Pareto distribution \cite{} were introduced. To avoid crowded model comparison and to aim our goal fitting the whole range of the data, we are going to consider only three models of one component model above. Additionally, each of distributions represent one, two, and three parameter model respectively.

\begin{figure}
\includegraphics[width=\maxwidth]{figure/sample_histogram-1} \caption[sample histrogram]{sample histrogram}\label{fig:sample histogram}
\end{figure}




	\subsection{Two-componenet models} 


Most general way to combine two distributions is to using mixture distribution. Mixed exponenetial could be a typical case. 

\begin{equation}
  \label{mixed_exp.pdf}
  f(x ; \omega, \lambda_1, \lambda_2) = \frac {\omega} {\lambda_1} e^{-x/\lambda_1} + \frac {1-\omega} {\lambda_2} e^{-x/\lambda_2}, \quad x > 0, \quad \lambda_1, \lambda_2 > 0, \quad \omega \in [0,1], 
\end{equation}

Alternatively, hybrid model could be considered, which is a distribution combining two different distributions at particular threshold. These hybrid approaches have more attention on the extreme value of daily rainfall by adopting heavy-tailed distribution in the tail part, like generalized Pareto, because some researches find out it has heavy tail behavior \cite{koutsoyiannis2004statistics}. From the precedent studies, the performance of hybrid models defeat the one of mixture models in genereal \cite{li2012simulation, furrer2008improving}. Hence, we mainly cover the hybrid models to be compared rather than other mixture distributions. 

\begin{equation}
  \label{gamma_gp_hybrid.pdf}
  f(x ; \alpha, \beta, \xi, \sigma, \theta) = f_{gamma}(x ; \alpha, \beta) I(x \leq \theta) + [1-F_{gamma}(\theta ; \alpha, \beta)] f_{GP}(x ; \xi, \sigma, \theta) I(x > \theta)
\end{equation}

At the threshold $\theta$, to make it continuous, a constraint $f(\theta-) = f(\theta+)$ is needed, which yields that the scale parameter $\sigma$ of generalized Pareto distribution is expressed as the reciprocal of the gamma hazard funciton, 

\begin{equation}
  \label{sigma_expression}
  \sigma = \frac {1 - F_{gamma}(\theta ; \alpha, \beta)} {f_{gamma}(\theta ; \alpha, \beta)}
\end{equation}

Thus the number of paramters in this model reduces from five to four. But still the selection problem of the threshold $theta$ remains without any guaranteed selection method. To pass away such huddle, \cite{li2012simulation} sugested another type of hybrid model, combination of exponential and generalized Pareto distributions.

\begin{equation}
  \label{exponential_gp_hybrid.pdf}
  f(x ; \lambda, \xi, \sigma, \theta) = \frac {1} {1 + F_{exp}(\theta ; \lambda)} [f_{exp}(x ; \lambda) I(x \leq \theta) + f_{GP}(x ;  \xi, \sigma, \theta) I(x > \theta)]
\end{equation}

With same constraint above, $f(\theta-) = f(\theta+)$, the number of parameters can be reduced again. 

\begin{equation}
  \label{theta_expression}
  \theta = -\lambda \ln{\frac {\lambda} {\sigma}}
\end{equation}

Especially this model outstands by deriving the threshold $\theta$ analytically.

%%%
%========================
\section{Phase-type distribution class}
%========================




%%%
%========================
\section{USHCN Texas daily precipitation data set}
%========================

From \cite{li2012simulation}, various kind of models were used to fit each stations of USHCN Texas daily precipitation data set. We chose five models among them to compare with PH model; Exponential, Gamma, kappa, hybrid of gamma and generalized Pareto(GGP), hybrid of exponential and generalized Pareto(EGP). 

For GGP model, its performance is affected by the choice of threshold value. Too large threshold would estimate its tail short by taking large emphasis on gamma distribution. Otherwise too small threshold set large emphasis on genralized Pareto and estimate its tail heavy. Thus we set threshold value used in GGP model moderately as 60\% quantile of each sample.

Overall, single component models; exponential, gamma, kappa, show inferior performances than other multi component models. With particular sample station(ID2), Exponential and gamma model underestimate tail thickness, besides kappa model overestimates it. Even with any other stations, gamma tends to exhibit short tail and kappa tends to exhibit heavy tail. 

\begin{figure}
  \centering
  \includegraphics[width = 15cm]{QQplots_ID2.pdf}
  \caption{QQ plots of ID2 modelled by each distributions}
\end{figure}

With this sample, GGP, EGP, and PH models show visually similar performances. Since the ultimate goal is to find general method generating stochastic daily precipitation model, it it essential to compare each methods with differenct cases. Furthermore, not only with qq plot, AIC \cite{akaike1974new} and BIC \cite{schwarz1978estimating} were used to do numerical comparison.

\begin{figure}
  \centering
  \includegraphics[width = 15cm]{QQplot_multicomp.pdf}
  \caption{QQ plots for 3 different stations modelled by GGP distribution(lest), EGP distribution(middle), and the PH distribution(right)}
\end{figure}







%%%
%========================
\section{World data}
%========================

In the research of \cite{papalexiou2012entropy}, wordwide GHCN data set was analyzed to figure out its shape of empirical distribution and tail behavior. To capture the shape characteristics of each sample, L-moments ratio was used. A typical L-moments ratio diagram is made with L-skewness versus L-kurtosis, but with precipitation data, but we prefer L-varaiation versus L-skewness as \cite{papalexiou2012entropy} done before. 



\begin{figure}
  \centering
  \includegraphics[width = 15cm]{lmrd.pdf}
  \caption{L-variation vs. L-skewness plot}
\end{figure}


\begin{figure}
\includegraphics[width=\maxwidth]{figure/bell_J_texas-1} \caption[Histograms of bell-shape(right), J-shape(middle), and one of USHCN Texas(right) data]{Histograms of bell-shape(right), J-shape(middle), and one of USHCN Texas(right) data}\label{fig:bell J texas}
\end{figure}





%%%
%========================
\section{Conclusion}
%========================


%following selects the referencing style
\bibliographystyle{./natbib}
%following selects the reference bib file
\bibliography{./myrefs.bib}

\end{document}
